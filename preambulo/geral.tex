\RequirePackage[l2tabu,orthodox]{nag} % Old habits die hard

\newskip\Otherbigskipamount   \Otherbigskipamount=15pt plus 5pt minus 5pt
\newcommand{\Otherbigskip}{\vspace{\Otherbigskipamount}}

\newcommand{\papersizeswitch}[3]{\ifnum\strcmp{\papersize}{#1}=0#2\else#3\fi}
\papersizeswitch{b5paper}{\def\classfontsize{12pt}}{\def\classfontsize{14pt}}

\documentclass[\classfontsize,\papersize,openany,showtrims,extrafontsizes]{memoir}
\RequireXeTeX

%\cftpagenumbersoff{chapter} %Para suprimir página dos capítulos no ToC

\showtrimsoff
\papersizeswitch{b5paper}{
    % Layouts dos papéis
    \pagebv
    \setlrmarginsandblock{26mm}{20mm}{*}
    \setulmarginsandblock{35mm}{30mm}{*}
    \setheadfoot{8mm}{10mm}
    \setlength{\headsep}{7mm}
    \setlength{\marginparwidth}{18mm}
    \setlength{\marginparsep}{2mm}
}{
    \papersizeswitch{a4paper}{
        \pageaiv
        \setlength{\trimtop}{0pt}
        \setlength{\trimedge}{\stockwidth}
        \addtolength{\trimedge}{-\paperwidth}
        \settypeblocksize{634pt}{448.13pt}{*}
        \setulmargins{4cm}{*}{*}
        \setlrmargins{*}{*}{0.66}
        \setmarginnotes{17pt}{51pt}{\onelineskip}
        \setheadfoot{\onelineskip}{2\onelineskip}
        \setheaderspaces{*}{2\onelineskip}{*}
    }{
    }
}
\ifnum\strcmp{\showtrims}{true}=0
    % Para imprimir B5 em um A4 com marcas de corte (trimmarks)
    \showtrimson
    \papersizeswitch{b5paper}{\stockaiv}{\stockaiii}
    \setlength{\trimtop}{\stockheight}
    \addtolength{\trimtop}{-\paperheight}
    \setlength{\trimtop}{0.5\trimtop}
    \setlength{\trimedge}{\stockwidth}
    \addtolength{\trimedge}{-\paperwidth}
    \setlength{\trimedge}{0.5\trimedge}
    
    % maior todos se houver trim marks
    \setmarginnotes{10pt}{95pt}{\onelineskip}

    \trimLmarks
    
    % coloca o jobname no canto esquerdo superior das marcas (trim mark)
    \renewcommand*{\tmarktl}{%
      \begin{picture}(0,0)
        \unitlength 1mm
        \thinlines
        \put(-2,0){\line(-1,0){18}}
        \put(0,2){\line(0,1){18}}
        \put(3,15){\normalfont\ttfamily\fontsize{8bp}{10bp}\selectfont\jobname\ \
          \today\ \ 
          \printtime\ \ 
          Page \thepage}
      \end{picture}}

    % Remove a marca (trim mark) centra para um layout mais limpo
    \renewcommand*{\tmarktm}{}
    \renewcommand*{\tmarkml}{}
    \renewcommand*{\tmarkmr}{}
    \renewcommand*{\tmarkbm}{}
\fi

\checkandfixthelayout                 % Checa se há erros no formato do papel

\usepackage{import}

\sideparmargin{outer}

\newcommand{\numsecao}{\arabic{section}} %para armazenar apenas a numeração da section

%
\usepackage{array}
\usepackage{amsmath,amssymb}
\usepackage[thmmarks,amsmath]{ntheorem}
\usepackage{thmtools}
\usepackage{bm}

% Link e url

\usepackage[hyphens]{url}             % Permite hifenização em URL's
\usepackage{nameref}
\usepackage[unicode=true,psdextra]{hyperref}
\usepackage{cleveref}
\crefname{appsec}{Apêndice}{Apêndices}

\usepackage{listings}                 % Pacote para listar códigos em LaTeX

% Imagens e Cores
\usepackage{graphicx}                 % Incluir imagens e usar cores
\usepackage[x11names]{xcolor}                   % Definir mais cores
\usepackage{eso-pic}                  % Marca d'água e mais
\usepackage{preambulo/coresdoc}
\graphicspath{{imagens/}}

\usepackage{type1cm}
\usepackage{lettrine}
\usepackage{GoudyIn}

\usepackage{csquotes}       % para quotes
\usepackage{indentfirst}    %para não identar primeiro parágrafo após seção, capítulo, etc.

% Tabela de conteúdos, Sumário, Table of contents (TOC)
\setcounter{tocdepth}{1}              % Profundidade da ToC
\setcounter{secnumdepth}{2}           % Profundidade na numeração de seções
\setcounter{maxsecnumdepth}{3}        % Profundidade máxima da numeração de seções

% Estilo do capítulo
%-----------------------------------------------
%- MyChapterStyles
%-----------------------------------------------
\makeatletter
\makechapterstyle{mychapterstyle}{
    \chapterstyle{default}
    \def\format{\normalfont\sffamily}

    \setlength\beforechapskip{0mm}

	\renewcommand*\thechapter{\arabic{chapter}}
    \renewcommand*{\chapnamefont}{\format\HUGE}
    \renewcommand*{\chapnumfont}{\format\fontsize{54}{54}\selectfont}
    \renewcommand*{\chaptitlefont}{\format\fontsize{42}{42}\selectfont}

    \renewcommand*{\printchaptername}{\chapnamefont\MakeUppercase{\@chapapp}}
    \patchcommand{\printchaptername}{\begingroup\color{dtugray}}{\endgroup}
    \renewcommand*{\chapternamenum}{\space\space}
    \patchcommand{\printchapternum}{\begingroup\color{dtured}}{\endgroup}
    \renewcommand*{\printchapternonum}{%
        \vphantom{\printchaptername\chapternamenum\chapnumfont 1}
        \afterchapternum
    }

    \setlength\midchapskip{1ex}

    \renewcommand*{\printchaptertitle}[1]{\raggedleft \chaptitlefont ##1}
    \renewcommand*{\afterchaptertitle}{\vskip0.5\onelineskip \hrule \vskip1.3\onelineskip}
	\renewcommand{\partnamefont}{\Large\scshape\centering\MakeLowercase}
	\renewcommand{\partnumfont}{\Large\scshape\centering\MakeLowercase}
	\renewcommand{\midpartskip}{\par\rule{1in}{0.5pt}\par}
	\renewcommand{\printparttitle}{\HUGE\scshape\centering\MakeLowercase}
}
\makeatother
%\chapterstyle{mychapterstyle}

%---------------------------------------------------------
%- VZ34
%---------------------------------------------------------

\usepackage{calc}
\newif\ifNoChapNumber
\makeatletter
\makechapterstyle{VZ34}{
  \renewcommand\chapternamenum{}
  \renewcommand\printchaptername{}
  \renewcommand\printchapternum{}
  \renewcommand\chapnumfont{\huge\bfseries}
  \renewcommand\chaptitlefont{\large\bfseries\raggedleft}
  \renewcommand\printchaptertitle[1]{%
    \begin{tabular}{@{}p{\textwidth-1cm-2em-4\tabcolsep }|!{\quad}p{1cm}}
	  \chaptitlefont ##1 & \ifNoChapNumber\relax\else\chapnumfont \thechapter\fi
    \end{tabular}
%    \begin{tabular}{@{}p{1cm}|!{\quad}p{\textwidth-1cm-2em-4\tabcolsep }}
%	  \ifNoChapNumber\relax\else\chapnumfont \thechapter\fi 
%	  & \chaptitlefont ##1
%    \end{tabular}
    \NoChapNumberfalse%
  }
  \renewcommand\printchapternonum{\NoChapNumbertrue}
}

%------------------- JP01 -------------------------

%---------------------------------------------------------
%- VZ34
%---------------------------------------------------------

\newif\ifchapternonum
\makeatletter
\makechapterstyle{JP01}{
  \renewcommand\chapternamenum{}
  \renewcommand\printchaptername{}
  \renewcommand\printchapternum{}
  \renewcommand\chapnumfont{\huge\bfseries}
  \renewcommand\chaptitlefont{\large\bfseries\raggedleft}
  \renewcommand\printchaptertitle[1]{%
	\noindent%
	\ifchapternonum%
	\begin{tabularx}{\textwidth}{X}%
		{\parbox[b]{\linewidth}{\chaptitlefont ##1}%
		\vphantom{\raisebox{15pt}{\chapnumfont 1}}}
	\end{tabularx}%
	\else
	\begin{tabularx}{\textwidth}{Xl}
		{\parbox[b]{\linewidth}{\chaptitlefont ##1}}
		& \chapnumfont \thechapter%
	\end{tabularx}%
	\fi
}
  \renewcommand\printchapternonum{\NoChapNumbertrue}
}


%---------------------------------------------------

%\chapterstyle{VZ34}
\chapterstyle{JP01}

\let\Part\part
\renewcommand\part[2][]{{\let\newpage\relax
\ifx\relax#1\relax\Part{#2}\else\Part[#1]{#2}\fi}}  % Definido partes

% Para textos com mais de uma coluna na página
\usepackage{multicol}

%Para itens, subitens... listas
\usepackage{enumitem}


% Cabeçalho e notas de rodapé
\def\hffont{\sffamily\small}
\makepagestyle{myruled}
\makeheadrule{myruled}{\textwidth}{\normalrulethickness}
\makeevenhead{myruled}{\hffont\thepage}{}{\hffont\leftmark}
\makeoddhead{myruled}{\hffont\rightmark}{}{\hffont\thepage}
\makeevenfoot{myruled}{}{}{}
\makeoddfoot{myruled}{}{}{}
\makepsmarks{myruled}{
    \nouppercaseheads
    \createmark{chapter}{both}{shownumber}{}{\space}
    \createmark{section}{right}{shownumber}{}{\space}
    \createplainmark{toc}{both}{\contentsname}
    \createplainmark{lof}{both}{\listfigurename}
    \createplainmark{lot}{both}{\listtablename}
    \createplainmark{bib}{both}{\bibname}
    \createplainmark{index}{both}{\indexname}
    \createplainmark{glossary}{both}{\glossaryname}
}
\pagestyle{myruled}
\copypagestyle{cleared}{myruled}      % Ao usar \cleardoublepage, utilize myruled ao invés de empty
\makeevenhead{cleared}{\hffont\thepage}{}{} % Remove leftmark em páginas brancas

\makeevenfoot{plain}{}{}{}            % Sem numeração de página em páginas pares no estilo plain (capítulo inicia)
\makeoddfoot{plain}{}{}{}             % Sem numeração de página em páginas ímpares no estilo plain (capítulo inicia)

% estilos de fontes para \*section, \*paragraph
\setsecheadstyle              {\huge\sffamily\raggedright}
\setsubsecheadstyle           {\LARGE\sffamily\raggedright}
\setsubsubsecheadstyle        {\Large\sffamily\raggedright}
%\setparaheadstyle             {\normalsize\sffamily\itseries\raggedright}
%\setsubparaheadstyle          {\normalsize\sffamily\raggedright}


% Hypersetup
\hypersetup{
    pdfauthor={\autordoc{}},
    pdftitle={\titulodoc{}},
    pdfsubject={\subtitulodoc{}},
    pdfdisplaydoctitle,
    bookmarksnumbered=true,
    bookmarksopen,
    breaklinks,
    linktoc=all,
    plainpages=false,
    unicode=true,
    colorlinks=false,
    citebordercolor=dtured,           % cor para links na bibliografia
    filebordercolor=dtured,           % cor de links de arquivos
    linkbordercolor=dtured,           % cor para links internos (mude a cor do quadro com linkbordercolor)
    urlbordercolor=s13,               % cor para links externos
    hidelinks,                        % para não mostrar quadros ou links coloridos.
}
% Próximas linhas são para fazer link correto ao usar pdfbookmark. O comportamento normal liga-se logo abaixo do título do capítulo. Este hack coloca o link logo acima do capítulo como qualquer outro uso normal do \chapter.
% Outra solução pode ser encontrada em http://tex.stackexchange.com/questions/59359/certain-hyperlinks-memoirhyperref-placed-too-low
\makeatletter
\renewcommand{\@memb@bchap}{%
  \ifnobibintoc\else
    \phantomsection
    \addcontentsline{toc}{chapter}{\bibname}%
  \fi
  \chapter*{\bibname}%
  \bibmark
  \prebibhook
}
\let\oldtableofcontents\tableofcontents
\newcommand{\newtableofcontents}{
    \@ifstar{\oldtableofcontents*}{
        \phantomsection\addcontentsline{toc}{chapter}{\contentsname}\oldtableofcontents*}}
\let\tableofcontents\newtableofcontents
\makeatother

% Confidencial
\newcommand{\confidentialbox}[1]{
    \put(0,0){\parbox[b][\paperheight]{\paperwidth}{
        \begin{vplace}
            \centering
            \scalebox{1.3}{
                \begin{tikzpicture}
                    \node[very thick,draw=red!#1,color=red!#1,
                          rounded corners=2pt,inner sep=8pt,rotate=-20]
                          {\sffamily \HUGE \MakeUppercase{Confidential}};
                \end{tikzpicture}
            }
        \end{vplace}
    }}
}

% Prefrontmatter
\newcommand{\prefrontmatter}{
    \pagenumbering{alph}
    \ifnum\strcmp{\confidential}{true}=0
        \AddToShipoutPictureBG{\confidentialbox{10}}   % uma caixa de confidencial em 10% ao fundo de cada página
        \AddToShipoutPictureFG*{\confidentialbox{100}} % uma caixa de confidencial em 100% ao fundo na primeira página
    \fi
}

\newcommand{\frieze}{%
    \AddToShipoutPicture*{
        \put(0,0){
            \parbox[b][\paperheight]{\paperwidth}{%
                \includegraphics[trim=130mm 0 0 0,width=0.9\textwidth]{frise-02}
                \vspace*{2.5cm}
            }
        }
    }
}

%\setlength{\parindent}{2em}

\setsecheadstyle{\sethangfrom{\noindent ##1}\Large\centering}  %Definir estilo e centralizar seções.

\usepackage{draftwatermark}

\SetWatermarkText{www.jeanpimenta.com \\[8pt] CC BY-SA 4.0}
\DraftwatermarkOptions{
    color={[gray]0.85},
    angle=45,
    hpos=0.5\paperwidth,vpos=0.5\paperheight,
    scale=0.4
}


\usepackage{tabularx}
\usepackage{changepage} %Para ajustar textos, tabelas, etc com relação às margens de texto

%Para criação de um glossário
\usepackage{makeidx} 
\makeindex 