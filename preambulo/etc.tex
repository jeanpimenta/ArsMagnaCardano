% Especificação de pacotes.

\usepackage{blindtext}
\usepackage{algorithm}
\usepackage{algpseudocode}
%\usepackage{pgfplots}                 % Ferramentas para plot
%\usetikzlibrary{
%    arrows,
%    matrix,
%    positioning,
%    shapes,
%    topaths,
%}
%\pgfplotsset{compat=1.7}

% Listings
\lstset{
    basicstyle=\footnotesize\ttfamily,% tamanho da letra utilizado na inserção de códigos
    breakatwhitespace=false,          % define se as quebras automáticas só devem acontecer com espaço em branco
    breaklines=true,                  % define quebra de linha automática
    captionpos=b,                     % define a posição de legenda para baixo
    commentstyle=\color{s14a},        % estilo de comentários
    deletekeywords={},                % caso queira excluir palavras-chave de um idioma
    escapeinside={\%*}{*)},           % caso queira addicionar LaTeX em seu código
    frame=single,                     % adiciona um quadro ao redor do código
    keywordstyle=\bfseries\ttfamily\color{s09}, % estilo da keyword
    language=Python,                  % linguagem do código
    morekeywords={*,...},             % caso queira adicionar mais keyword para o set
    numbers=left,                     % onde colocar a numeração das linhas; opções possíveis (none, left, right)
    numbersep=5pt,                    % distância entre numeração das linhas e o código
    numberstyle=\sffamily\tiny\color{dtugray}, % estilo usado para numeração das linhas
    rulecolor=\color{dtugray},        % caso não definido, a cor do quadro pode ser alterada em quebras de linhas
    showspaces=false,                 % mostra espaços em todos os lugares adicionando sublinhados; substitui os 'showstringspaces'
    showstringspaces=false,           % sublinhar espaços apenas dentro de strings
    showtabs=false,                   % mostra tabs dentro de strings adicionando sublinhados
    stepnumber=1,                     % passo entre numeração de linhas. Se for 1, cada linha será numerada
    stringstyle=\color{s07},          % estilo literal da string
    tabsize=2,                        % define o padrão de tabulação para 2 espaços
    title=\lstname,                   % mostra o nome do arquivo dos arquivos incluídos com \lstinputlisting
}