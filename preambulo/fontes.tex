% Fontes para textos (http://www.macfreek.nl/memory/Fonts_in_LaTeX)
% Utilize XeLaTeX
\usepackage{fontspec}
%%%%%%%%%%%%%%%%%%%%%%%%%%
%%%%%%%%%%%%%%%%%%%%%  1
\usepackage{xunicode}
\defaultfontfeatures{Ligatures=TeX}
\setmainfont[Mapping=tex-text,Numbers=OldStyle]{Junicode}
%\setmainfont[Ligatures={TeX,Rare}]{Hoefler Text}
%\usepackage{xspace}
%%%%%%%%%%%%%%%%%%%%%  1
%%%%%%%%%%%%%%%%%%%%%%%%%%
\newfontfamily\junicode[Mapping=tex-text,Numbers=OldStyle]{Junicode}
\newcommand\que{\junicode\char"E8BF\xspace\normalfont}
\newcommand\nb{\"{\junicode\char"0273}\xspace\normalfont}
\newcommand\ee{\junicode\char"0119\normalfont}
\newcommand\ct{\junicode\char"EEC5\normalfont}
\newcommand\utraco{\junicode\char"016B\normalfont}
\newcommand\rdois{\junicode\char"A776\normalfont}
\newcommand\ft{\junicode\char"E8DF\normalfont}
\newcommand\ff{\junicode\char"EBA6\normalfont}
\newcommand\jfi{\junicode\char"EBA2\normalfont}
\newcommand\aeft{\ae\ft}
%%%%%%%%%%%%%%%%%%%%%%%%%%
%%%%%%%%%%%%%%%%%%%%%%%%%%
%%%%%%%%%%%%%%%%%%%%%  2
%\defaultfontfeatures{Ligatures={TeX}}
%\setmainfont{TeXGyrePagella}
%\fontspec[Ligatures={Rare}]{Hoefler Text}
%\setmainfont[Mapping=tex-text,Ligatures={Common, Rare}]{Hoefler Text}
%[Mapping=tex-text]
%Com TexGyrePagella e Hoefler Text
%%%%% Com TexGyrePagella e Hoefler Text
%Abaixo deve ser a abreviação de -que, uma espécie de q-sz
%\newcommand{\que}{\rareliga{}fs{}\normalfont}

% Linguagem
\usepackage{polyglossia}    % tipografia multilinguística e hifenizações apropriadas
\setmainlanguage{brazil}
\setotherlanguage[variant=classic,babelshorthands]{latin}

%Para utilização de sinais gráficos sob ou sobre a letra
%\usepackage{wsuipa}
\usepackage{ipa}