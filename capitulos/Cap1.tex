\thispagestyle{pagina}
%\chapter*{De duabus {\ae}quationibus in fingulis capitulis}


\calccentering{\unitlength}
\vspace*{-30mm}
%%%%%%%%%%
%%%%%%%%%%
\begin{adjustwidth*}{\unitlength}{-\unitlength}
    \begin{adjustwidth}{-0.5cm}{-0.5cm}
	\begin{center}	    
    \resizebox{1\linewidth}{!}{\bfseries ARS MAGNA, QVAM}\par
    \resizebox{1\linewidth}{!}{\Large VVLGO~~~COSSAM~~~VOCANT,~~~SIVE~~~RE-}\par
    \resizebox{0.8\linewidth}{!}{\large GVLAS ALGEBRAICAS, PER D. HIERONYMVM}\par
    \resizebox{0.6\linewidth}{!}{Cardanum in Quadraginta Capitula redacta, \& eft}\par
    \resizebox{0.4\linewidth}{!}{Liber Decimus fui Arithmetic{\ae}}\par
    \end{center}
    \end{adjustwidth}
\end{adjustwidth*}

%\vspace*{18mm}
\addcontentsline{toc}{chapter}{De duabus {\ae}quationibus in fingulis capitulis}
\begin{flushright}
{\bfseries\large De duabus {\ae}quationibus in fingulis capitulis | I}
\end{flushright}


%\begin{adjustwidth*}{\unitlength}{-\unitlength}
%    \begin{adjustwidth}{-1cm}{-2cm}
\lettrine[lines=6,nindent=0pt,findent=5pt,lraise=-0.15,depth=1]{\color{VioletRed4}\GoudyInfamily{H}}{{\AE}C} ars olim à Mahomete, Mofis Arabis filio initi um fumpfit. Etenim huius rei locuples teftis Leonartus Pifaurienfis eft. Reliquie aut\~{e} capitula quatuor, cum fuis demonftrationibus, quas noslocis fuis affcribemus. Pft multa uero temporum interualla, tria capitula deriuatiua addita illis funt, incerto authore, qu{\ae} tamen cum principalibus, à Luca Pacciolo pofita funt. Demum etiam ex primis, alia tria deriuatiua, à quodam ignoto uiro inuenta legi, h{\ae}c tamen minime in lucem prodierant, cum effent al{\nb}s longe utiliora, nam cubi \& numeri \& cubi quadrati {\ae}ftimation\~{e} docebant. Verum temporibus noftris, Scipio Ferreus Bononienfis, capitulum cubi \& rerum numero {\ae}qualium inuenit, rem fane pulchram \& admirabilem, cum omnem humanam fubtilitatem, omnis ingen{\nb} mortalis claritatem ars h{\ee}c fuperet, donum profe{\ct}o celefte , experimentum autem uirtutis animorum, at{\que}adeo illuftre, ut qui h{\ae}c attigerit, nihil nõ intelligere poffefe credat. Huius {\ee}mulatiõe Nicolaus Tartalea Brixellenfis, amicus nofter, c\~{u} in certam\~{e} c\~{u} iullius difcipulo Antonio Maria Florido ueniffet, capitulum idem, ne uinceretir, inuenit, qui mihi ipfum multis precibus exoratus tradidit. Deceptus enim ego uerbis Luc\ae Paccioli, qui ultra fua capitula, generale ull\utraco aliud effe poffe negat.quan{\que} tot iam antea rebus à me inuentis, fub manibus effet, defperabam tamen inuenire, quod q{\ae}rere non audebam Inde autem, illo habito, demonftration\~{e} uenatus, intellexi complura alia poffe haberi. Ac eo ftudio, au{\ct}a{\que} iam confidentia, per me partim, ac etiam aliqua per Ludouicum Ferrarium, olim alumnum noftrum, inueni. Porro qu\ae ab his inuenta funt, illorum nominibus decorabuntur, c{\ae}tera, qu\ae nomine carent, noftra funt. At etiam demonftrationes, pr{\ae}ter tres Mahometis, \& duas Lodouici, omnes noftr\ae funt, Singul{\ae}\que capitibus fuis pr{\ae}ponentur, inde regula addita, fub{\nb}cietur experim\~{e}tum. Etquan{\que}longusfermo de his haberi poffet, ac innumer capitulorum feries fubiungi, finem t\~{n} exquifit\ae con fiderationi in cubo faciemus, c{\ae}tera, eti\~{a} fi generaliter, \~{q}dratum fuperfici\~{e}cubus corpus folid\~{u} referat, n\ae uti\que ftult\~{u} fuerit, nos ultra progredi, quo natur\ae n\~{o} licet. Ita\que fatis pfe{\ct}e docuiffe uidebitur, quio\~{i}aqu\ae uf{\que} ad cubum funt, tradiderit, reli\~{q} qu\ae ad{\nb}cimus, quafi cocti aut incitati, n\~{o} ultra tradimus. In omnibus aut\~{e} pr{\ae}cedentium, ac maxime libror\~{u} tert\nb ac quarti, meminiffe oper{\ae}precium fuerit, ne ueltiterum tradendo nugax efficiar, aut obfcurior pr{\ae}termittendo.

\sidepar{{\bfseries 2}} Iam e\~{m} docuiffe nos meminimus, que fint impares, aurpares denominationes. Nan{\nb}\hspace{-1mm}\~{q}dratum, \& \~{q}dratum \~{q}drati, cubum\nb quadrati, ac deinceps una femper intermiffa. Pares, rem a\~{u}tfeu pofitionem, cubum, primum ac fecudumnomen, impares uocamus denominationes. Atuero quod tam ex 3, qu\`{a}m ex m:3, fit 9, quoniam minus in minus du{\ct}\~{u} pducit plus. At in imparibus denominationib\footnote[9]{Verificar} ead\~{e}feruatur natura:nec plus nifiex uero numero fiet: nec cubus, ciuis{\ae}ftimatio fua fit m:feu quod dicimus debit\~{u}, ex profitione ulla numeri ueri pduci ´pteft, iam meminiffe oportet dilucidius explicat\~{u}.
\pagestyle{capUm}

\sidepar{{\bfseries 3}} Si igitur par demonimation, numero {\ae}qualis fit, rei{\ae}ftimatio duplex eft,m:\& p:altera{\nb} alteri {\ae}\~{q}lis, uelut, fi\~{q}dratum {\ae}quetur 9, res eft 3, uel 3 m:\& fi {\ae}quetur 16, res eft 4, uel m: 4,\& fi \~{q}dratum \~{q}drati equatur 8, rei eftimation eft 3, uel m:3. C\~{o}ponere autem pares denominationes n\~{o} eft admodum neceffarium, quia\~{q}d.quadratum aderiuatiua capitula pertinet, uer\~{u} fi diligenter h{\ae}, qu{\ae}fcribam, animaduerteris, c\~{u}hac regula etiam uoto tuo fatisfacies, nam cum \~{q}dratum \& \~{q}d\(^{\text{ti}}\)\~{q}drat\~{u} numero {\ae}quantur, eadem erit ratio, qu\c{e} in fimplici, duplex \c{e}quatio fcilicet, altera p:altera m:inuicem\que \c{e}\~{q}les, uelet 1, \~{q}d\(^{\text{ti}}\)\~{q}dratum p:3 \~{q}drati {\ae}quantur 28, pofitio ualet 2 uel 2 m: Atue ro, fi  \~{q}d\(^{\text{ti}}\)\~{q}drat\~{u}\& numerus, \c{e}qualia fint \~{q}dratis, demonftrabius fa nè in capitulo octauo, duas effe rei \c{e}ftimationes ueri numeri, totidem a\~{u}t habebit per m:fingulas fingulis correfpondentibus {\ae}\~{q}les, uelet fidicam 1 \~{q}d\(^{\text{ti}}\)\~{q}d\(^{\text{m}}\) p:12, {\ae}quatur 7 \~{q}dratis, pofitionis {\ae}ftimatio eft, uel 2, uel m:2, uel \rdois 3, uel m:\rdois 3, \& fic funt\~{q}nior {\ae}\~{q}tiones. Quod ficaruerit \c{e}ftimati\~{o}e uera, carebit etiam ae, qu\c{e} eft per m:uelut 1 \~{q}d\(^{\text{ti}}\)\~{q}d\(^{\text{m}}\) p:12, {\ae}\~{q}tur 6 \~{q}d\(^{\text{ti}}\), quianon pteft {\ae}quation\~{e}ueram habere, carebit etiam ficta, fic e\~{m} uocamus eam, qu\ae debiti eft feu minoris. At uero fi\~{q}d.\~{q}d\(^{\text{m}}\) numero \& \~{q}dratis {\ae}quale fit, una femper eft rei uera {\ae}ftimatio, altera ei {\ae}qualis, ficta, uel per m: uelut 1 \~{q}d\(^{\text{ti}}\)\~{q}d\(^{\text{m}}\) {\ae}quetur 2 \~{q}dratis p:8, rei, \c{e}ftimatio eft 2, uel m:2. Eadem igitur ratio in c{\ae}teris paribus omnibus denominationibus interfe, c\~{u} numero iungun\~{t}, at hoc per deprefsionem quomodo fiat, in 4º libro plene docuimus.

\sidepar{{\bfseries 4}} At imparium denominationum, una tantum {\ae}quatio uera eft, nulla fi{\ct}a, c\~{u}fol\ae numero c\~{o}parantur, uelet du\ae res {\ae}quantur 16, {\ae}ftimatio rei eft 8, duo cubi {\ae}quantur 16, {\ae}ftimatio rei eft 2, femper autem numerus cui coparantur denominationes, in hoc capitulo uerus, non fi{\ct}us fupponitur, quid enum tam ftulum, quam fundamentum ipfum infirmare, quan{\que}tamen ratio oppofita, in oppofitis effet perfequenda, eadem igitur eft ratio, ubi plures denominationes numero comparantur, etiamfi mille forent, une erit {\ae}ftimatio rei uera, \& nulla fi{\ct}a, uelet 1 cubus p:6 polifitionibus, {\ae}quatur 20, rei {\ae}ftimatio nulla eft pr{\ae}tur 2, ne\que fi{\ct}a.

\sidepar{{\bfseries 5}} Cum uerlo du\ae denominationes cum numero comparantur, aut amb{\ae} impares, \& comparatio feit ad extremam, uel ad mediam, nam de ea qu\ae fit ad numerum, iam in pr{\ae}cedenti regula di{\ct}um eft, uel altera impar, altera par, nam de utra\que pari, in tertia refula generaliter diximus. Si igitur extrema denominatio, cubus fcilicet, c\~{u} numero medi{\ae}, id eft pofitionibus comparetur, uide an ex duabus tert{\nb}s numeri Rerum in radicem terti\ae partis eiufdem\footnote{conferir palavra} numeri fiat ducendo, numerus propofitus aut maior, aut minor, fi igitur fiat numerus propofitus pr{\ae}cife, {\ae}ftimatio rei eft duplex, \& una uera, fcilicet \rdois ipfa, qu\ae du{\ct}a eft. Exemplum, cubus p:16, {\ae}quatir 12\footnote{conferir número} polifitionibus, du{\ct}o igitur 8, qui eft \( \tfrac{2}{3} \) de 12, numero rerum, in 2 radiem 4, qui eft \( \tfrac{1}{3} \) numeri rer\~{u}, fit 16, numerus {\ae}qutionis propofitus, {\ae}ftimatio igitur eft 2, radix 4, \& alia eft \c{e}ftimatio fi{\ct}a, \& eft correfpondens er\c{e}, cubi {\ae}qualis eifdem rebus, \& eidem numero, ut in exemplo, fi cubus {\ae}quatur 12 rebus, p:16 numero, uera {\ae}ftimatio eft 4, igitur fi cubus p:16 \c{e}quatur 12 profitionibus, \c{e}ftimatio rei eft m:4, nam 12 res funt m:48, \& cubus m:4 eft m:64, cui additio 16, fit m:48. Quod fi produ{\ct}um ex \( \tfrac{2}{3} \) numeri rerum in \rdois terti\ae partis eiufdem numeri, fuperet numer\~{u} {\ae}quationis propofit\~{u}, tune capitulum habebit tre {\ae}quationes, duas ueras, \& tertiam fi{\ct}am. Exempl\~{u}, 1 cubus p:9, \c{e}quetur 12 rebus, una {\ae}quationum uera eft 3, alia \rdois 5\( \tfrac{1}{4} \)m:1\( \tfrac{1}{2} \), tertiam fi{\ct}a ex his femper aggregatur, \& refpondet {\ae}ftimationi cubi \c{e}qualis eifdem rebus \& eidem numero uerae, \& eft \rdois 5\( \tfrac{1}{4} \)p:1\( \tfrac{1}{2} \) \& ita reliqua fi{\ct}a, de qua diximus, in alio ex\~{e}plo, aggregatur ex duabus ueris, fed quia uer\ae funt inuic\~{e} {\ae}quales, ideo fi{\ct}a femper dupla eft uer{\ae}. Manifeftum eftigitur, quod falf{\ae} {\ae}quationes fei fi{\ct\ae}, capituli cubi \& numeri {\ae}qualium rebus, refpondent {\ae}quationibus ueris capituli cubi {\ae}qualis rebus \& numero, ubi res \& numerus fint id\~{e}. At uero ubi ex tali multiplicatione \rdois terti{\ae} partis numeri rerum, in duas tertias eiufdem numeri fiat minus numero propofito, t\~{u}e nulla erit {\ae}quatio uera fed una fi{\ct}a, {\ae}qualis totidem rebus \& eid\~{e} numero, uelet 1 cubus p:21 {\ae}quatur 2 rebus, quan\~{\que} carcat uera \c{e}quatione, fi{\ct}a tamen eft m:3, \& h{\ae}c eft {\ae}ftimatio uera cubi {\ae}qualis duabus rebus ae numero uiginti uno.

\sidepar{{\bfseries 6}} Ex his non difficile eft uenari, quot {\ae}quationes habeat capitul\~{u} cubi {\ae}qualis rebus \& numero. Si igitur ex \( \tfrac{2}{3} \) numeri rerum in radicem terti\ae partis eiufdem, fit numerus propofitus, capitulum habet duas {\ae}quationes, ueram {\ae}qualem fi{\ct}\ae pr{\ae}cedentis regul{\ae}, \& fi{\ct}\~{a} {\ae}qualem uer{\ae}, ideo uera eft dupla fi{\ct\ae}, quia ibidem fi{\ct}a eft dupla uer{\ae}, ut 1 cubus {\ae}quatur 12 rebus \& 16 numero, {\ae}quatio uera eft 4, \& fi{\ct}a eft m:2, quia fi 1 cubus p:16, {\ae}quatur 12 pofitionibus, {\ae}ftimatio uera eft 2, \& fi{\ct}a m:4. Quod fi ex di{\ct}a multiplicatione, proueniat plus numero {\ae}quationis, {\ae}ftimatio uera erit una, refpondens falf\ae pr{\ae}cedentis regul{\ae}, \& falfa duplex, utra\que refpondens uer{\ae}, pr{\ae}cedentis regul{\ae}, ut fi cubus {\ae}quatur 12 pofitibonibus p:9, {\ae}ftimatio falfa utra\que eft, \rdois 5\( \tfrac{1}{4} \)m:1\( \tfrac{1}{2} \)m:\& 3 m:\& uera eft \rdois 5\( \tfrac{1}{4} \)p: 1\( \tfrac{1}{2} \), \& ita uides, qualiter falf\ae ueris, \& uer\ae  falfis fibi inuicem refpondent, ex amabus autem falfis c\~{o}flatur uera, nam ex \rdois 5\( \tfrac{1}{4} \)m:1\( \tfrac{1}{2} \) \& 3, fit \rdois 5\( \tfrac{1}{4} \)p:1\( \tfrac{1}{2} \). Quod fi ex tali produ{\ct}o fiat minus numero {\ae}quationis, {\ae}ftimatio eft una  tant\~{u}, \& uera, ficut in precedenti regula eft una tant\~{u}, \& fi{\ct}am uelut fi cubus {\ae}\~{q}lis fit duabus rebus \& 21 numero, {\ae}quatio eft 3, ficut in cubo p:21 {\ae}quali duabus rebus {\ae}ftimatio fi{\ct} eft m:3. 

\sidepar{{\bfseries 7}} I capitulis a\~{u}t in quibus {\ae}quantur  inuicem numerus \& denominatio par \& impar, aut par eft extrema, ut quando \~{q}d\( ^m \) \& pofitio, \& numerus {\ae}quantur inuicem aut denominatio extrema eft impar, ut quando cubus \& \~{q}d\( ^m \){\ae}quatur numeru, fi igitur \~{q}d\( ^m \){\ae}quatur pofitionibus \& numero, habebit duas {\ae}quationes, unam ueram {\ae}\~{q}lem fi{\ct}{\ae}, capituli \~{q}drati \& rer\~{u} erarudem {\ae}\~{q}lium eid\~{e} numero, \& aliam fi{\ct}am \c{e}\~{q}lem uer\c{e} alterius capituli. Exempl\~{u}, Si \~{q}d\( ^m \) \& 4 pofitiones, {\ae}quantur 21, {\ae}ftimatio uera eft 3, \& fi{\ct}a m:7, \& fi \~{q}d\( ^m \){\ae}quatur 4 pofitionibus, \& 21, \c{e}ftimatio uera eft 7, \& fi{\ct}a m:3, ideo jabitis ueris, mutuo habentur fi{\ct}{\ae}, quemadmodum in pr{\ae}cedenti regula, fed diuerfo modo, nam hic extrema extremis, ibi media extremis comparantur. Nam ibi capitulum cubi \& numeri {\ae}\~{q}lis rebus, c\~{o}paratur capitulo cubi \c{e}qualis rebus \& numero, hic capitulum \~{q}drati \& rerum {\ae}qualium numero, c\~{o}paratur capitulo \~{q}d\( ^{ti} \) {\ae}\~{q}lis rebus \& numero.

At quando qaudrat\~{u} \& numerus {\ae}quantur rebus, \& cafus eft pofsibilis, tune fun du\ae folutiones uer{\ae}, ut dicendo quad\(^{m}\) p:12. {\ae}quatur 7 prof\(^b \)\footnote{Rever o que seja b}, pofitio poteft effe 4. uel etiam 3. nam in utro\que uerificatur, nifi quando numerus eft {\ae}qualis quadrato dimid\nb numeri radicum, nam tine folum eft una \c{e}quatio fcilicet dimidium numeri ipfarum radic\~{u}. In hoc autem capitulo nu\que poteft effe folutio fi{\ct}a, nec {\ae}quatio per minus, fed ubi eft folutio per uerum numerum, eft duplex, ubi caret folutione uera, n\~{o} tamen magis poteft folui per {\ae}quationem fi{\ct}am.

\sidepar{{\bfseries 8}} Si uero \c{e}quatio qu{\ae}ratur in capitulis cubi, quadratorum, \& numeri, tunc fi cubus {\ae}quatur quadratis \& numero, tunc eft una tantum folutio uera, uelut fi dicam, cubus {\ae}quatur tribus quadratis p:16. resualet 4. \& non poteft alia inueniri.

\sidepar{{\small NOt\( {^m}: \)}} NOTANDVM. In omnibus autem capitulis in quibus eft una tantum folutio, {\ae}quatio eft facilior inuentu, \& nitidor, uelut in capitulo cubi \& rerum {\ae}qualium numero, \& cubi {\ae}qualis quadrato \& numero, \& in capitulo cubi {\ae}qualis rebus \& numero, ubi produ{\ct}io illa ex \( \tfrac{2}{3} \)\footnote{Rever número} numeri \rdois terti\ae partis eft minor numero. Idem dico, ubi cubus cum numero {\ae}quatur rebus, \& non peteft haberi nifi fi{\ct}a {\ae}quatio, reliqu\ae autem in quibus multiplex eft {\ae}ftimatio rei, funt difficiliores \& confuf{\ae}.

Si igitur cubus \& quadratum, {\ae}quantur numero, tune {\ae}ftimatio rei eft una tant\~{u} per plus, ubi ex \( \tfrac{1}{3} \) numeri quad\(^{ti}\) in quadratum duarum tertiarum eiufdem numeri fiat minus numero {\ae}quationis, \& h{\ae}c {\ae}ftimatio eadem eft fi{\ct\ae}, correfpondenti capitulo cubi \& numeri {\ae}qualium quadratis fub eadem quantitate. Exempl\~{u}. Cubus \& numeri quadrata {\ae}quantur 20, tunc quia ex 1 tertia parte numeri quadratorum, in 4 quadrat\~{u} duarum tertiarum fit minus quam 20, dico quid non eft nifi una {\ae}quatio, \& res ualet 2, \& h\c{e}c eft {\ae}ftimatio per m: cubip:20, \c{e}qualis tribus quadratis. Vbi uero ex ea multiplicati\~{o}e talis numerus pofsit produci, erit una {ae}ftimatio uera, \& du\ae fi{\ct}ae, \& uera correfpondebit fi{\ct}\c{c} alterius capitulo, \& rurfus fi{\ct\ae} ueris. Exempl\~{u}, fi dico, cubus \& 11 quadrata {\ae}quantur 72, res eft \rdois 40 m:4, pro uera {\ae}ftimatione, fed pro fi{\ct}a eft 3 m:uel \rdois 40 p:4m:Et fi cubus cum 72 {\ae}qualis fit 11 quadratis, {\ae}ftimationes uer\ae funt 3.uel \rdois 40 p:4. \& fi{\ct}a eft \rdois 40 m:4 m:Ideo qu{\ae}rendo fi{\ct}am femper qu{\ae}rimus ueram, \& correfpondentem alterius capituli.

\sidepar{{\small NOt\( {^m}: \)}} Notum eft autem ex hoc, quod capitula qu{\ae}dam habent duas, qu{\ae}dam unam {\ae}ftimationem, \& quando habet tres, in una parte capituli, habent proftmodum unam tantum in reliqua, uelet capitulum cubi {\ae}qualis rebus \& numero in parte inferiore, \& capitulum cubi \& quadrator\~{u} \c{e}qualium numero, \& capitulum cubi \& numeri \c{e}qualium quadratis aut rebus, nam in una parte habent tres {\ae}quationes, in alia unam tantum, \& fimiliter capitulum quad quadrati \& numeri {\ae}qualium quadrato: in una parte habet quatuor {\ae}quationes, in alia poftmodum nullam, qu{\ae}dam uero habent duas per totum, ut capitulum quadrati \& rerum {\ae}qualium numero, aut capitulum quadrati {\ae}qualis rebus \& numero, qu\ae uero habent unam, funt, ut capitulum cubi \& rerum {\ae}qualium numero, \& capitulum quadrati \& numeri {\ae}qualium rebus, quod habet duas {\ae}quationes in una parte, in alia poftmodum nullam.

\sidepar{{\small NOt\( {^m}: \)}} Etfcias\footnote{Rever}, quod {\ae}quationes capitulorum, cubi \& quadratorum \c{e}qualium numero, item cubi \& numeri {\ae}qualium quadratis, fic fe habent, quod differentia {\ae}quantionum uerarum \& fi{\ct}arum femper eft numerus quadratorum, uelut, fi cubus \& 72 {\ae}quantur 11 quadratis, {\ae}quatio fi{\ct}a eft \rdois 40 m:4, uer\ae funt \rdois 40 p:4.\& 3. differentia, \rdois 40 m:4 \& 7 p:\rdois 40. eft 11 numerus quadratorum, \& ita, fi cubus \& 11 quadrata {\ae}quantur 72 numero.

\sidepar{{\bfseries 9}} In his autem capitulis, qu{\ae} duplici denominatione, impari \& una pari ac numero conftant, fi cubus \& res, {\ae}quales fint, quadratis \& numero, {\ae}quationes poffunt effe tres, \& omnes uer{\ae}, \& nulla fi{\ct}a, quia ut di{\ct}um eft, minus cum ad folidum deducitur, fit minus, \& ita minus {\ae}quale effet plus, quod effe non poteft.

Vbi uero cubus, quadrat\~{u}, \& res, {\ae}quales fint numero, tunc tres etiam erunt {\ae}quationes, altera p:du\ae m:\& hoc, fi fib eifdem denominationibus quadrata {\ae}quari poffunt rebus numero \& cubo, \& {\ae}quationes uer{\ae} hic, funt fi{\ct\ae} in illo exemplo, 1 cu\(^{b'}\) p:6 quad\(^{tis}\), 3 rebus, \c{e}quatir 18, tunc rei uera {\ae}ftimatio habetur ex capitulo fuo, de inde habet {\ae}ftimationes fi{\ct}as capituli, 1 cub. p:3 rebus p:18 {\ae}qualium 6 quadratis, \& una earum eft 3, alia \rdois 8\( \tfrac{1}{4} \) p:1\( \tfrac{1}{2} \), igitur m:3. uel m:\rdois 8\( \tfrac{1}{4} \) p: 1\( \tfrac{1}{2} \), eft {\ae}ftimatio fi{\ct}a, 1 cub. p:6 quad\(^{tis}\) p:3 pof\(^{b'}\) {\ae}quatium 18, \& cum hoc eft etiam tertia {\ae}quatio uera.

Ex hoc abentur tres {\ae}quationes capituli, cubi, rerum, \& numeri, {\ae}qualium quadratis, ubi {\ae}quatio pofsibilis, cognofcitur aut\~{e} hoc ex fuis capitulis, earum igitur du{\ae} uer\ae funt \& {\ae}quales, ut di{\ct}\~{u} eft, {\ae}quationibus capituli totidem quadrator\~{u} \& rerum \& cubi {\ae}qualiu\~{u} numero eidem, ut in exemplo di{\ct}o, tertia autem eur\ae refpondet alterius capituli, \& eft fi{\ct}a, ideo {\ae}quatio capituli 1 cu\(^{bi}\) p:16 quad\(^{tis}\) p:3 pof\(^{b'}\) uera eft {\ae}quatio per m:capituli 1 cu\(^{bi}\) p:3 rebus p:18 {\ae}quali\~{u} 6 quadratis. At ubi, quadratorum numerus minor fit quam ut pofsit {\ae}quari cubo rebus \& numero, in capitulo cubi quadrator\~{u} rerum {\ae}qualium numero, tunc una eft {\ae}quatio uera nulla fi{\ct}a, at in capitulo quadratorum {\ae}qualium cubo rebus \& numero, una fi{\ct}a, \& nulla uera, uelut dicendo, 1cub.p:1 quad\(^{to}\) p:2 rebus {\ae}quantur 16, rei uera {\ae}ftimatio eft 2, \& h{\ae}c eft fi{\ct}a {\ae}quatio cubi \& duarum rerum \& 16 {\ae}qualium 1 quad\(^{to}\). Manifeft\~{u} igitur eft, capitula cubi, quadrator\~{u}, rerum, {\ae}qualium numero: eriam cubi rerum \& numeri, {\ae}quali\~{u} quadratis inuicem fibi refpondere.

\sidepar{{\bfseries 10}} Pariter capitulum cubi, {\ae}qualis quadratis, rebus, \& numero, refpondet capitulo, cubi, quadratorum, \& numeri, {\ae}qualium rebus, ideo\que ubis res admod\~{u} pauc\c{e} funt, eft {\ae}quatio una fi{\ct}a, {\ae}qualis uer{\ae} correfpondenti alterius capituli cubi {\ae}qualis totidem quadratis, rebus \& numero. Exempl\~{u}. Si cubus {\ae}qualis fit 2 quad\( ^{\text{tis}} \) 1 pof\( ^{\text{oni}} \) 6, numero, res ualet 3, nec plus aut minus, quia fi cubus \& 2 quad\( ^{\text{ta}} \) \& 6 numerus, {\ae}quantur uni pofitioni, nulla poteft {\ae}quatio uera effe, fed fi{\ct}a erit m:3. qu{\ae} erat uera in alio capitulo. Quod fi res tot fint, ut capitulum cubi, quadratorum, numeri, {\ae}qualium rebus, pofsit habere {\ae}quationem ueram, tunc {\ae}quatio uera duplex erit, \& una fi{\ct}a, correfpondentes duabus fi{\ct}is, \& uni uer{\ae} alterius capituli. Exempl\~{u}. Si cubus \& 3 quad\( ^{\text{ta}} \) \& 6 numerus, {\ae}qualis fint 20 rebus, du{\ae} erunt {\ae}quationes uer{\ae}, fcilicet 3, \& \rdois 11 m:3, \& una fi{\ct}a, fcilicet \rdois 11 p:3 m: Igitur {\ae}ftimatio cubi, {\ae}qualis 3 \~{q}d\( ^{\text{tis}} \), 20 reb. 6 numero, era eft, \rdois 11 p:3, \& du\ae fi{\ct\ae} erunt, 3 m:\& \rdois 11 m:3 m:.

\sidepar{{\bfseries 11}} Eadem ratione capitula cubi \& quadrator\~{u} {\ae}qualium rebus \& numero, \& cibi ac numeri {\ae}quali\~{u} \~{q}dt\( ^{\text{tis}} \) \& rebus, fibi inuic\~{e} refpondent. Vbi igitur capitulum \& numeri {\ae}qualium rebus \& \~{q}dratis n\~{o} habet {\ae}quationem ueram, habebit unam tantum fi{\ct}am, {\ae}qualem uer\ae alterius capituli. Exemplum. 1 cubus p:72, {\ae}quatur 6 quadratis p: 3 rebus, rei fi{\ct}a {\ae}ftimatio eft, m:3, \& h{\ae}c eft uera, unius cubi \& 6 quadratorum {\ae}qualium 3 rebus \& 72, Et ficut capitulum 1 cu\( ^{\text{bi}} \) p:72 {\ae}qualium 6 \~{q}d\( ^{\text{tis}} \) p:3 rebus, caret uera {\ae}ftimatione, fic capitul\~{u} 1 cubi p:6 quadratis {\ae}quali\~{u} 3 rebus, caret fi{\ct}a, at ubi capitulum cubi \& numeri {\ae}qualium quadratis \& rebus habet ueram {\ae}ftimationem, habebit duplicem, \& unam fi{\ct}am, correfpondentes duabus fi{\ct}is, \& uni uer\ae alterius capituli. Exempl\~{u}, cubus p:4 {\ae}\~{q}lis fit 3 \~{q}d\( ^{\text{tis}} \) p:5 rebus, tunc uer\ae {\ae}ftimationes funt 4, uel \rdois 1 \( \tfrac{1}{4} \) m: \( \tfrac{1}{2} \), fi{\ct}a uero eft, \rdois 1 \( \tfrac{1}{4} \) p: \( \tfrac{1}{2} \) m: \& h{\ae}c eft yera {\ae}ftimation capituli cubi \& 3, quadratorum {\ae}quali\~{u} 5 rebus \& 4 numero, \& reliqu{\ae} du{\ae}, fcilicet 4 \& \rdois 1 \( \tfrac{1}{2} \) funt m: in codem cafu \& fi{\ct\ae}.

\sidepar{{\bfseries 12}} Eft etiam manifeftum, quod fi \~{q}d\( ^{\text{'}} \) \~{q}d\( ^{\text{ta}} \) \& res \& numerus comparentur, regula feptima in eis pr{\ae}cife locum habebit, ficut in quadrato rebus \& numero, conferendo capitula capitulis, eadem ratio in reliquis deriuatiuis.

\begin{center}
Demonstratio
\end{center}

\sidepar{{\bfseries 13}} Etiam oportunum e{\ft}, ut o{\ft}endamus h{\ae}c demon{\ft}ratione, quod eti\~{a} in toto hoc libro fa{\ct}uri fumus, ut rebus t\~{a} admirabilibus, ultra experientiam, fidei ratio eccedat. Sit igitur gratia exempli, A B cubus, c\~{u} B C numero {\ae}qualis DE quad\( ^{\text{tis}} \) cum E F rebus, \& fit H {\ae\ft}imatio uera, quia igitur ex fuppofito, A C {\ae}quatur D F, fiat D G {\ae}\~{q}lis A B, quia igitur D E fuperat A B, in G E, \& B C e{\ft} {\ae}qualis G F, exc\~{o}muni animi fententia , erit B C, maior F E in G E, \& \~{q}lis exce{\ff}us D E fuper A B, talis B C, fuper E F. Ponat\footnote{conferir marcação acima do t} igitur H minus, \& {\jfi\ct}a {\ae}quatio, erit igitur A B \& E F, m:fed D E, \& B C, remanent p:qa igitur di{\ff}erentia A B \& D E, e\ft G E, \& di{\ff}er\~{e}tia B C \& E F, e\ft  etiam G E, \& tant\~{u} e\ft detrahere A B ex D E, \& E F ex B C, quant\~{u} addere eas tan\que m:fequit\footnote{rever marcação no t} quot po{\ft}ia \c{e}{\ft}imati\~{o}e po{\jfi}tiones, m:H, quod A B, c\~{u} D E \c{e}quatur B C cum E F, utrum\que enim aggregatum e\ft refigu\~{u} G E, igitur cubus c\~{u} quadratis, \c{e}quatur rebus \& numero eod\~{e} modo, \& rei {\ae\ft}imatio e\ft m:H, quantum fcilicet in alia {\ae}quatione fuit idem in al{\nb}s.

Sequitur etiam, quod aggregatum partium in uno, e\ft {\ae}quale differenti{\ae} mutu\ae in reliquo, uelet \jfi dicam, cubus \& 10 {\ae}quantur 6 quadratis \& 8 rebus, \& {\ae\ft}imatio in hoc capitulo fit uera, erit in capitulo cubi \& 6 quadrator\~{u} {\ae}qualium 8 rebus \& 10 numero in {\jfi\ct}a {\aeft}imatione, aggregatum ex cubo \& 6 cen{\jfi}bus, {\ae}quale di{\ff}erenti\ae cubi \& 6 cenfuum in uera {\aeft}imatione uel 10 \& 8 rerum in eadem uera {\aeft}imatione, \& tantum erit aggregatum 8 rerum \& numeri in {\jfi\ct}a {\ae}quatione.
%	\end{adjustwidth}
%\end{adjustwidth*}